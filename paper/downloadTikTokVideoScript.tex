\documentclass{article}

\usepackage{datetime}
\usepackage{fancyhdr}
\usepackage{graphicx}
\graphicspath{{./image/}}
\usepackage[colorlinks=true,linkcolor=blue]{hyperref}
\usepackage{lastpage}

\pagestyle{fancy}
\lhead{Small script to download TikTok video}
\rhead{\currenttime\ \today}
\rfoot{\LaTeX\space by Quan, Tran Hoang}

\begin{document}
\author{Quan, Tran Hoang}
\title{Small script to download TikTok video}
\maketitle{}

\tableofcontents{}\newpage

\part{Introduction}
\section{About TikTok}
TikTok is a video sharing social networking service. It is used to create short dance, lip-sync, comedy and talent video\cite{WikipediaTikTok}.
\section{About Quan, Tran Hoang}
At the time this document is being wrote, Quan is a Computer Science freshman at the VNUHCM - University of Science\cite{aboutquan}\cite{aboutquanblog}.
\section{Why I wrote this script}
However, since TikTok's policies does not allow users to download videos, downloading a video is a hard, time-wasting process. Sometimes you see a nice (rarely) video on TikTok, wanna post it on your Instagram or your Facebook story but cannot download them. Sharing it, then your friends just see a thumbnail instead of the original video.
\\\\
Knowing this, I created a small script to download TikTok videos, which can easily done by a mouse click. This script will match the video link from the website and output it, so you can do whatever you want with it.
\section{Is this illegal?}
de jure\footnote{by law} illegal, de facto\footnote{in reality} legal.

\part{The Script}
\section{Source code}\label{codeexplain}
The script is originally written in JavaScript, correctly VanillaJS\cite{vanillajs}. Here's the source code:

\begin{verbatim}
/**
 * Create a bookmark and add these codes to it.
 * Open the video, then click the bookmark to get the download link.
 * Or just simply paste these code to web console to get download link, too.
 * Remember to delete these comments before adding code to bookmark!
 *
 * Code by @trhgquan - https://github.com/trhgquan
 */

javascript:var video_url = document.getElementsByTagName('video');
prompt('video_url', video_url[0].src);
\end{verbatim}
\textbf{Explanation:} This script will get all \verb|<videos>| tags and return the first \verb|<video>| \verb|src| aka \verb|video_url[0].src|. This is the active video link.

\section{Download}
\subsection{Link}
You can fork the file, download it or embed it here:
\\
\href{https://gist.github.com/trhgquan/35e2758208fe79c62f997519551e27a3}{https://gist.github.com/trhgquan/35e2758208fe79c62f997519551e27a3}

\subsection{Copy raw}\label{copyraw}
Follow these steps to get the raw version without downloading it.
\begin{enumerate}
\item Go to  \href{https://gist.github.com/trhgquan/35e2758208fe79c62f997519551e27a3}{https://gist.github.com/trhgquan/35e2758208fe79c62f997519551e27a3}\cite{scriptlink}
\item Choose \textbf{Raw} button (as shown below).
\\
\includegraphics[scale=0.25]{rawbtn}
\item The raw code appears, copy it.
\end{enumerate}

\part{Using Script}
Before we get start, I have to mention that this script works on \textbf{every} browsers on \textbf{every} operating systems.
\\\\
There are two ways to use this script,

\begin{itemize}
    \item \textbf{The bookmarklet way:} for newbie that didn't get use to the Developer Tools
    \item \textbf{The console way:} for pros and Firefox user.
\end{itemize}

\section{Using Script with bookmarklet.}\label{bookmarklet}
\begin{enumerate}
\item Copy the code (recommend take \ref{copyraw} first).
\item Create a new bookmark. This step is different depends on your browser.
\item Paste the code to the URL field. This step is important, because if you paste the code wrong way e.g missing the "javascript:" prefix, the code will not work.
\item To use: open TikTok video on browser, click on the newly-created bookmark. The video's link will be on a prompt box and you can copy it.
\end{enumerate}

\section{Using Script with the Developer Tools}
The same as \ref{bookmarklet}, with some modify:
\begin{enumerate}
\item Copy the code (recommend take \ref{copyraw} first).
\item Open TikTok video on browser.
\item Open Developer Tool, choose \textbf{Console}.
\item Paste the code to the console.
\item The video's link will now appears in the console, the same time a prompt box appears just like \ref{bookmarklet}. You can choose link in both box since it's the same link.
\end{enumerate}

\part{Further}
\section{Notice}
This code can get at least and at most 1 (one) video at a time. To download multiple videos, you must open each video at one time, then trigger the code to get the link.
\section{Practice}
Uncomfortable with downloading 1 video each time, and you got hundreds of them? Actually this is a practice for you: write your own TikTok video downloader, using my script as an example.
\\\\
\underline{\textbf{The idea:}} since there are many \verb|<video>| tags in the whole document (\ref{codeexplain}), a loop through the selector variable may help. At that time, look for the download link and have suitable actions \textit{e.g} create cookie, create CSRF token,..etc.
\\\\
Be creative and make it your own way. Cheers!

\medskip
\begin{thebibliography}{9}
\bibitem{WikipediaTikTok}
\textit{Wikipedia.org: TikTok}
\\\texttt{\href{https://en.wikipedia.org/wiki/TikTok}{https://en.wikipedia.org/wiki/TikTok}}
\bibitem{aboutquan}
\textit{Quan, Tran Hoang's Resume}
\\\texttt{\href{https://www.chumeochuixoong.blogspot.com/p/my-resume.html}{https://www.chumeochuixoong.blogspot.com/p/my-resume.html}}
\bibitem{aboutquanblog}
\textit{Quan, Tran Hoang's blog: About Me}
\\\texttt{\href{https://www.chumeochuixoong.blogspot.com/p/about-me.html}{https://www.chumeochuixoong.blogspot.com/p/about-me.html}}
\bibitem{vanillajs}
\textit{VanillaJS is a fast, lightweight, cross-platform framework
 for building incredible, powerful JavaScript applications.}
\\\texttt{\href{http://vanilla-js.com}{http://vanilla-js.com}}
\bibitem{scriptlink}
\textit{Quan, Tran Hoang: Simple way to download TikTok video, without using third-party apps / websites. }
\\\texttt{\href{https://gist.github.com/trhgquan/35e2758208fe79c62f997519551e27a3}{https://gist.github.com/trhgquan/35e2758208fe79c62f997519551e27a3}}
\end{thebibliography}

\end{document}