\documentclass[]{article}

\usepackage{fancyhdr}
\usepackage{float}
\usepackage{graphicx}
\usepackage[colorlinks=true,linkcolor=blue]{hyperref}
\usepackage{lastpage}
\usepackage{listings}
\usepackage[margin=1in]{geometry}
\usepackage[nottoc,notlot,notlof]{tocbibind}
\usepackage[utf8]{vietnam}
\usepackage{url}
\usepackage{xcolor}
\usepackage{booktabs}

\lstset{
    frame = single,
    basicstyle = \ttfamily,
    breaklines = true,
    basicstyle=\footnotesize\ttfamily,
}

\graphicspath{{./image/packetTracer/}}

\pagestyle{fancy}
\lhead{Báo cáo Đồ án 3 - Packet Tracer}
\rhead{Trường Đại học Khoa học Tự nhiên - ĐHQG HCM}
\lfoot{\LaTeX\ by \href{https://github.com/trhgquan}{Quan, Tran Hoang}.}

% Title Page
\title{Báo cáo Đồ án 3 - Packet Tracer}
\author{Nhóm sinh viên lớp 19CTT2}
\date{Mùa Xuân năm 2021}

\renewcommand\arraystretch{1.5}

\begin{document}
\maketitle
\tableofcontents
\listoftables
\pagebreak

\section{Thông tin}
\subsection{Nhóm sinh viên thực hiện}
\begin{table}[ht]
    \centering
    \caption{Các sinh viên thực hiện Đồ án 3 - Packet Tracer.}
    \begin{tabular}[t]{lcl}
        \toprule
        Họ \& tên&MSSV&Vai trò\\
        \midrule
        Trần Hoàng Quân&19120338&Cài đặt Lab 1;\\
        &&Cài đặt Lab 3;\\
        &&Viết báo cáo;\\
        Lâm Hải Triều&19120407&Cài đặt Lab 2;\\
        &&Quay video demo;\\
        &&Screenshot cho báo cáo;\\
        \bottomrule
    \end{tabular}
\end{table}

\subsection{Phần mềm sử dụng}
Nhóm sử dụng \href{https://www.netacad.com/courses/packet-tracer}{phần mềm Cisco Packet Tracer} \texttt{phiên bản 7.3.1}. Giao diện phần mềm có thể khác với các phiên bản cũ hơn.

\section{Lab 1}
\subsection{Cách cài đặt}
Để cài đặt Lab 1, nhóm đã thực hiện theo các bước sau:
\begin{enumerate}
\item Chọn Router, Switch và PC, sau đó nối dây theo đề bài.
\begin{figure}[H]
    \centering
    \includegraphics[scale=0.5]{Lab01-hinh1.png}
    \caption{Thực hiện cài đặt các thành phần của mạng và đi dây}
\end{figure}
\item Thực hiện cấu hình Router bằng các lệnh đã cung cấp.
\begin{itemize}
\item Chọn \texttt{Router > Config}
\item Load nội dung config như đề bài đã cho:
\begin{lstlisting}
Router(config)# int fa0/0
Router (config-if)#ip address 192.168.10.1 255.255.255.0
Router (config-if)#exit
Router (config-if)#ip dhcp pool dhclab
Router(dhcp-config)# network 192.168.10.0 255.255.255.0
//network network-number [mask | /prefix-length]
Router(dhcp-config)#default-router 192.168.10.1
Router(dhcp-config)#dns-server 192.168.10.2
Router(dhcp-config)# exit
Router(config)# ip dhcp excluded-address 192.168.10.1 192.168.10.10
//not in the dhcp pool
Router(config)#ip dhcp excluded-address 192.168.10.248 192.168.10.254
//not in the dhcp pool
Router(config)#exit
Router(config)#wr //set the all PC network configuration to DHCP
\end{lstlisting}
\begin{figure}[H]
    \centering
    \includegraphics[scale=0.5]{Lab01-hinh2.png}
    \caption{Load config của Router.}
\end{figure}
\end{itemize}
\item \texttt{Truy cập từng PC > Desktop > IP Configuration}. Tick phần \texttt{DHCP} để chỉnh IP sang IP cấp phát động.\label{checkDHCP}
\begin{figure}[H]
    \centering
    \includegraphics[scale=0.5]{Lab01-hinh3.png}
    \caption{Chỉnh IP mỗi máy từ Static sang DHCP}
\end{figure}
\end{enumerate}

\subsection{Kiểm thử}
\subsubsection{Kiểm tra xem IP các máy có được cấp phát DHCP hay không}
Từ thao tác \ref{checkDHCP}, nếu phần IP có câu \texttt{DHCP request successful} thì ta có thể xác nhận IP của máy được cấp phát DHCP. IP được cấp phát trong bài sẽ nằm trong khoảng \texttt{192.168.10.1} đến \texttt{192.168.10.10} và trong khoảng \texttt{192.168.10.248} đến \texttt{192.168.10.254}.
\begin{figure}[H]
    \centering
    \includegraphics[scale=0.5]{Lab01-hinh4.png}
    \caption{Xác nhận trạng thái cấp phát IP của mỗi máy}
\end{figure}
\subsubsection{Kiểm thử xem các PC có thể tương tác với nhau hay không}
Ta có thể dùng thao tác \texttt{ping} để kiểm tra kết nối giữa các máy. Tại mỗi PC, vào \texttt{Desktop > Command Prompt}. Từ đây, ta nhập lệnh \texttt{ping <ip máy nhận>}. Nếu các gói tin ICMP được gửi thành công, ta có thể xác nhận hai máy đã kết nối.
\begin{figure}[H]
    \centering
    \includegraphics[scale=0.5]{Lab01-hinh3.png}
    \caption{Kết quả ping giữa các máy.}
\end{figure}

\subsection{Báo cáo}
Đề bài nêu 3 câu hỏi:
\begin{enumerate}
    \item \textbf{Địa chỉ IP sau khi được cấp phát DHCP} của PC0, PC1 và PC2?
    \item \textbf{Địa chỉ Gateway} của PC0, PC1 và PC2?
    \item Server nào là \textbf{DNS Server} của PC0, PC1 và PC2?
\end{enumerate}
Sau khi cài đặt hệ thống mạng, nhóm đã thu được các kết quả sau đây:
\begin{figure}[H]
    \centering
    \includegraphics[scale=0.5]{Lab01-hinh4.png}
    \caption{Thông tin IP PC0 - Lab1}
\end{figure}
\begin{figure}[H]
    \centering
    \includegraphics[scale=0.5]{Lab01-hinh5.png}
    \caption{Thông tin IP PC1 - Lab1}
\end{figure}
\begin{figure}[H]
    \centering
    \includegraphics[scale=0.5]{Lab01-hinh6.png}
    \caption{Thông tin IP PC2 - Lab1}
\end{figure}
Các kết quả có thể được tổng hợp lại trong bảng sau:
\begin{table}[H]
    \centering
    \caption{Kết quả các thử nghiệm ở Lab1.}
    \begin{tabular}[t]{lccc}
        \toprule
        PC&Địa chỉ IP&Gateway&DNS Server\\
        \midrule
        PC0&192.168.10.13&192.168.10.1&192.168.10.2\\
        PC1&192.168.10.13&192.168.10.1&192.168.10.2\\
        PC2&192.168.10.13&192.168.10.1&192.168.10.2\\
        \bottomrule
    \end{tabular}
\end{table}

\section{Lab 2}
\subsection{Cách cài đặt}
Để cài đặt Lab2, nhóm đã thực hiện theo các bước sau:
\begin{enumerate}
\item Chọn Router, Switch và PC theo mô hình đề bài, sau đó tiến hành nối dây.
\item Cấu hình IP Static cho các PC.
\item Cấu hình Router 1 chuyển hướng gói tin đến Router 2.
\item Cấu hình Router 2 chuyển hướng gói tin đến Router 1.
\end{enumerate}
\subsection{Kiểm thử}
\subsubsection{Kiểm thử các PC có thể tương tác với nhau hay không}
\begin{enumerate}
    \item Chọn 2 PC bất kỳ cùng mạng và ping đến PC còn lại.
    \item Chọn 2 PC bất kỳ khác mạng và ping đến PC còn lại.
\end{enumerate}
\subsection{Báo cáo}

\section{Lab 3}
\subsection{Cách cài đặt}
Để cài đặt Lab3, nhóm đã thực hiện theo các bước sau:
\begin{enumerate}
    \item Chọn Router, Switch và PC theo mô hình đề bài.
    \item Nối dây theo đề bài.
    \item Cấu hình Server HTTP và Server DHCP.
    \item Cấu hình Router 2 chuyển hướng gói tin DHCP và HTTP đến Router 1.
    \item Cấu hình Router 3 chuyển hướng gói tin DHCP và HTTP đến Router 1.
    \item Cấu hình cho 2 Router 2 và 3 giao tiếp với nhau.
\end{enumerate}
\subsection{Kiểm thử}
\subsubsection{Kiểm thử các PC đã được cấp phát DHCP hay chưa}
\subsubsection{Kiểm thử các PC có thể truy cập trang web hay chưa}
\subsubsection{Kiểm thử các mạng 172.16.1.0 và 172.16.2.0 có thể giao tiếp với nhau hay chưa}
\subsection{Báo cáo}
s

\begin{thebibliography}{10}
\bibitem{tranluixuong}
Trần Lùi Xuống
\end{thebibliography}
\end{document}