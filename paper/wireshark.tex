\documentclass[]{article}

\usepackage{fancyhdr}
\usepackage{float}
\usepackage{graphicx}
\usepackage[colorlinks=true,linkcolor=blue]{hyperref}
\usepackage{lastpage}
\usepackage{listings}
\usepackage[margin=1in]{geometry}
\usepackage[nottoc,notlot,notlof]{tocbibind}
\usepackage[utf8]{vietnam}
\usepackage{url}
\usepackage{xcolor}

\graphicspath{{./image/wireshark/}}

\lstset{
  language = Python,
  frame = single,
  basicstyle = \ttfamily,
  breaklines = true,
  basicstyle=\footnotesize\ttfamily,
  keywordstyle=\bfseries\color{green!40!black},
  commentstyle=\color{purple!40!black},
  identifierstyle=\color{blue},
  stringstyle=\color{orange},
}

\pagestyle{fancy}
\lhead{Báo cáo Đồ án Wireshark: Bắt và phân tích gói tin}
\rhead{Trường Đại học Khoa học Tự nhiên - ĐHQG HCM}
\lfoot{\LaTeX\ by \href{https://github.com/trhgquan}{Quan, Tran Hoang}.}

% Title Page
\title{Báo cáo Đồ án Wireshark: Bắt và phân tích gói tin}
\author{Sinh viên thực hiện: Trần Hoàng Quân - MSSV: 19120338}

\begin{document}
\maketitle
\tableofcontents
\pagebreak

\section{ICMP}
\subsection{Mục đích của việc ping?}
Lệnh \texttt{ping} giúp ta kiểm tra xem \textbf{có kết nối được với host hay không}, \textbf{khi gửi các gói tin có mất gói nào không} và \textbf{thời gian gửi/nhận các gói tin}.

\subsection{Có bao nhiêu gói tin trong quá trình ping?}
Có tổng cộng 8 gói tin, bao gồm 4 gói request (gửi đi) và 4 gói response (nhận về).
\begin{figure}[H]
\centering
\includegraphics[scale=0.4]{ICMP/ip.jpg}
\caption{Các gói tin ICMP bắt được, sử dụng Wireshark}
\end{figure}

\subsection{Địa chỉ MAC nguồn và địa chỉ MAC đích?}
\begin{itemize}
\item Địa chỉ MAC nguồn: \texttt{0a:00:27:00:00:10}
\begin{figure}[H]
\centering
\includegraphics[scale=0.4]{ICMP/MAC-host.jpg}
\caption{Lấy địa chỉ MAC và địa chỉ IP máy host bằng lệnh \texttt{ipconfig /all}}
\label{fig:macandiphost}
\end{figure}

\item Địa chỉ MAC đích: \texttt{08:00:27:e8:1e:0c}
\begin{figure}[H]
\centering
\includegraphics[scale=0.4]{ICMP/MAC-guest.jpg}
\caption{Lấy địa chỉ MAC và địa chỉ IP máy guest bằng lệnh \texttt{ipconfig /all}}
\label{fig:macandipguest}
\end{figure}

\item Địa chỉ MAC nguồn và đích capture được bằng Wireshark
\begin{figure}[H]
\centering
\includegraphics[scale=0.4]{ICMP/MAC.jpg}
\caption{Địa chỉ MAC nguồn và đích capture được bằng Wireshark}
\end{figure}
\end{itemize}

\subsection{IP nguồn và IP đích?}
Do ta đang \texttt{ping} từ máy host đến guest nên IP nguồn là của máy host, IP đích là của guest.
\begin{itemize}
\item IP nguồn: \texttt{192.168.38.2} (Xem hình \ref{fig:macandiphost} ở trang \pageref{fig:macandiphost})
\item IP đích: \texttt{192.168.38.1} (Xem hình \ref{fig:macandipguest} ở trang \pageref{fig:macandipguest})
\end{itemize}
Các IP capture được bằng Wireshark:
\begin{figure}[H]
\centering
\includegraphics[scale=0.4]{ICMP/IP.jpg}
\caption{Các IP capture được, sử dụng Wireshark}
\end{figure}

\subsection{Nội dung phần Data?}
Nội dung phần Data capture được là
\\
\texttt{Data: 6162636465666768696a6b6c6d6e6f7071727374757677616263646566676869 }
\begin{figure}[H]
\centering
\includegraphics[scale=0.4]{ICMP/data.jpg}
\caption{Data capture được, sử dụng Wireshark}
\end{figure}

\section{HTTP}

\subsection{Phiên bản HTTP của trình duyệt và của Server?}
\begin{itemize}
\item Trình duyệt: Phiên bản HTTP được gửi trong request header. Trình duyệt ở máy em dùng phiên bản HTTP/1.1
\begin{figure}[H]
\centering
\includegraphics[scale=0.4]{HTTP/browser-http.jpg}
\caption{Phiên bản HTTP được gửi trong request header}
\end{figure}
\item Server: Phiên bản HTTP được gửi trong response header. Server sử dụng phiên bản HTTP/1.1
\begin{figure}[H]
\centering
\includegraphics[scale=0.4]{HTTP/server-http.jpg}
\caption{Phiên bản HTTP được gửi trong response header}
\end{figure}
\end{itemize}

\subsection{Ngôn ngữ trình duyệt có thể chấp nhận Server?}
Ta tìm phần \texttt{Accept-Language}\cite{MSDN-Language} và thấy:
\\
\texttt{Accept-Language: vi,fr-FR;q=0.9,fr;q=0.8,en-US;q=0.7,en;q=0.6,ru;q=0.5\textbackslash r\textbackslash n }.
\begin{figure}[H]
\centering
\includegraphics[scale=0.4]{HTTP/browser-language-accept.jpg}
\caption{Ngôn ngữ trình duyệt có thể chấp nhận server, gửi trong request header}
\end{figure}

\subsection{IP máy của bạn và của server gaia.cs.umass.edu?}
Dùng Wireshark ta bắt được:
\begin{figure}[H]
\centering
\includegraphics[scale=0.4]{HTTP/http-ip.jpg}
\caption{Xác định các IP trong request bằng Wireshark}
\end{figure}

Theo đó:
\begin{itemize}
\item IP máy em: \texttt{10.0.137.80}
\item IP server: \texttt{128.119.245.12}
\end{itemize}

\subsection{Mã trả về từ server và ý nghĩa?}
Server trả về mã \texttt{200 OK}, nghĩa là request đã thành công.\cite{MSDN-200}
\begin{figure}[H]
\centering
\includegraphics[scale=0.4]{HTTP/error-code.jpg}
\caption{Mã \texttt{200 OK} cho biết request đã thành công.}
\end{figure}

\subsection{Thời điểm file HTML được chỉnh sửa lần cuối ở server?}
Thuộc tính \texttt{Last-Modified}\cite{MSDN-lastmod} nằm trong response header, ở đây là:
\\
\texttt{Last-Modified: Sat, 05 Dec 2020 06:59:02 GMT\textbackslash r\textbackslash n}
\begin{figure}[H]
\centering
\includegraphics[scale=0.4]{HTTP/last-modified.jpg}
\caption{Thời điểm sửa đổi lần cuối.}
\end{figure}

\subsection{Có bao nhiêu byte được trả về trong nội dung trình duyệt?}
Trong response 1306 có tổng cộng 126 bytes được trả về.
\begin{figure}[H]
\centering
\includegraphics[scale=0.4]{HTTP/byte-data.jpg}
\caption{Xem số lượng byte trả về.}
\end{figure}

\subsection{Kiểm tra request GET đầu tiên, em có tìm thấy dòng IF-MODIFIED-SINCE?}
Sử dụng Wireshark mở gói tin request GET đầu tiên. Ta nhận thấy, trong request GET đầu tiên không có dòng \texttt{IF-MODIFIED-SINCE}
\begin{figure}[H]
\centering
\includegraphics[scale=0.4]{HTTP/khongcoifmodified.jpg}
\caption{Không tìm thấy dòng \texttt{IF-MODIFIED-SINCE} trong GET request đầu tiên.}
\label{fig:khongmodified}
\end{figure}

\subsection{Kiểm tra nội dung của file, server có trả về nội dung rõ ràng không?}
Dựa theo hình \ref{fig:khongmodified} ở trang \pageref{fig:khongmodified}, nội dung server trả về rõ ràng, đầy đủ các phần đúng như browser request.

\subsection{Kiểm tra nội dung request GET thứ 2, em có tìm thấy dòng IF-MODIFIED-SINCE?}
Trong GET request thứ 2, file có dòng \texttt{IF-MODIFIED-SINCE}.
\begin{figure}[H]
\centering
\includegraphics[scale=0.4]{HTTP/coifmodifiedsince.jpg}
\caption{Tìm thấy dòng \texttt{IF-MODIFIED-SINCE} trong GET request thứ 2.}
\end{figure}
Phía sau dòng \texttt{IF-MODIFIDED-SINCE} là một ngày nhất định, ý nghĩa là truy vấn xem trang được request có thay đổi từ ngày đó hay không.

\subsection{Mã trả về từ server? Server có trả về file hay không?}
Server khi đó trả về mã \texttt{304 Not Modified}\cite{MSDN-304}, trong response chỉ có response header mà không có response body.
\begin{figure}[H]
\centering
\includegraphics[scale=0.4]{HTTP/noidung304.jpg}
\caption{Nội dung repsonse sau khi request có dòng \texttt{IF-MODIFIED-SINCE}.}
\end{figure}

Khi có thuộc tính \texttt{IF-MODIFIED-SINCE}, nếu trang không được chỉnh sửa sau ngày tháng được quy định trong \texttt{IF-MODIFIED-SINCE} thì server trả về trạng thái 304, báo rằng trang không có thay đổi. Khi nhận được trạng thái 304, trình duyệt sẽ load trang từ trong bộ nhớ cache.\cite{MSDN-modified}

\section{FTP}
\subsection{Có bao nhiêu loại gói tin FTP? Giải thích ý nghĩa từng gói?}
\begin{figure}[H]
\centering
\includegraphics[scale=0.4]{FTP/loaigoitin.jpg}
\caption{Các loại gói tin FTP}
\end{figure}
Ý nghĩa từng gói tin:\cite{packetinfo}
\begin{enumerate}
\item \texttt{AUTH}: Cơ chế xác thức / bảo vệ.
\item \texttt{CWD}: Đổi thư mục làm việc.
\item \texttt{DELE}: Xóa file.
\item \texttt{FEAT}: Xác định các chức năng mà server hỗ trợ.
\item \texttt{MLSD}: Liệt kê nội dung của một thư mục.
\item \texttt{PASS}: Xác thực mật khẩu (password).
\item \texttt{PASV}: Mở chế độ Passive.
\item \texttt{PWD}: Trả về thư mục đang hoạt động của host.
\item \texttt{RETR}: Nhận một bản copy của file.
\item \texttt{STOR}: Chấp nhận và lưu trữ dữ liệu dưới dạng file ở phía server.
\item \texttt{SYST}: Trả về kiểu hệ thống.
\item \texttt{TYPE}: Đặt kiểu truyền dữ liệu (Binary / ASCII).
\item \texttt{USER}: Xác thực tên người dùng (username).
\end{enumerate}
\subsection{IP nguồn, đích các gói tin FTP?}
Trong các request, IP nguồn là IP client, IP đích là IP server.
\begin{itemize}
\item IP nguồn: \texttt{192.168.38.2}
\item IP đích: \texttt{192.168.38.1}
\end{itemize}
Với các response thì ngược lại.
\begin{figure}[H]
\centering
\includegraphics[scale=0.4]{FTP/ipnguondich.jpg}
\caption{IP nguồn, đích của các gói tin}
\end{figure}

\subsection{MAC nguồn, đích các gói tin FTP?}
\begin{itemize}
\item Địa chỉ MAC nguồn: \texttt{0a:00:27:00:00:10}
\item Địa chỉ MAC đích: \texttt{08:00:27:e8:1e:0c}
\end{itemize}
Với các response thì ngược lại.
\begin{figure}[H]
\centering
\includegraphics[scale=0.4]{FTP/macnguondich.jpg}
\caption{MAC nguồn, đích của các gói tin}
\end{figure}

\subsection{FTP sử dụng port ở server và client là bao nhiêu?}
Để xem port mà server và client sử dụng, ta chọn mục \texttt{Transmission Control Protocol}.
\\\\
Trong mục này, \texttt{Source port} là port mà client đang dùng; \texttt{Destination port} là port mà server đang dùng.
\begin{figure}[H]
\centering
\includegraphics[scale=0.4]{FTP/portnguondich.jpg}
\caption{Port FTP của server và port FTP của client}
\end{figure}
Ở đây, server đang sử dụng port \texttt{21} và client đang sử dụng port \texttt{49937}.

\subsection{FTP sử dụng mode passive hay active?}
Xem các repsonse packet, ta có thể nhận thấy FTP đang sử dụng mode \texttt{passive}:
\begin{figure}[H]
\centering
\includegraphics[scale=0.4]{FTP/passive.jpg}
\caption{FTP sử dụng mode passive}
\end{figure}

\subsection{Chúng ta có lấy được username và password từ các gói tin FTP không?}
Nếu username hợp lệ, server mới yêu cầu password; trạng thái đăng nhập thành công được trả về trong một packet riêng, không bao gồm username và password.
\\\\
Nhìn chung, ta có thể lấy được username và password từ các gói tin rời rạc nhau.
\begin{figure}[H]
\centering
\includegraphics[scale=0.4]{FTP/thongtinuser.jpg}
\caption{Username và password được gửi qua các gói tin.}
\end{figure}

\begin{thebibliography}{10}
\bibitem{MSDN-Language}
Mozilla Developer Network.
\textit{Accept-Language},
\url{https://developer.mozilla.org/en-US/docs/Web/HTTP/Headers/Accept-Language}, truy cập \today

\bibitem{MSDN-200}
Mozilla Developer Network.
\textit{200 OK},
\url{https://developer.mozilla.org/en-US/docs/Web/HTTP/Status/200}, truy cập \today

\bibitem{MSDN-304}
Mozilla Developer Network.
\textit{304 Not Modified},
\url{https://developer.mozilla.org/en-US/docs/Web/HTTP/Status/304}, truy cập \today

\bibitem{MSDN-lastmod}
Mozilla Developer Network.
\textit{Last-Modified}
\url{https://developer.mozilla.org/en-US/docs/Web/HTTP/Headers/Last-Modified}

\bibitem{MSDN-modified}
Mozilla Developer Network.
\textit{If-Modified-Since},
\url{https://developer.mozilla.org/en-US/docs/Web/HTTP/Headers/If-Modified-Since}, truy cập \today

\bibitem{packetinfo}
SS64.com.
\textit{List of raw FTP commands},
\url{https://ss64.com/rawftp.html}, truy cập \today

\end{thebibliography}
\end{document}