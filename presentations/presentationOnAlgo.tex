\documentclass[12pt]{beamer}
\usepackage[utf8]{vietnam}
\usepackage{lmodern}
\usepackage{graphicx}
\usetheme{AnnArbor}

\begin{document}
    \author{Trần Lùi Xuống}
    \title{Thuật toán Floyd-Warshall}
    \subtitle{Tìm đường đi ngắn nhất giữa mọi cặp đỉnh}
    %\logo{}
    \institute{fit@hcmus}
    \date{mùa Xuân 2021}
    %\subject{}
    %\setbeamercovered{transparent}
    %\setbeamertemplate{navigation symbols}{}
    \begin{frame}[plain]
        \maketitle
    \end{frame}

    \begin{frame}
    \frametitle{Nội dung chính}
    \tableofcontents
    \end{frame}

    \begin{frame}[t]
        \frametitle{Bài toán đường đi ngắn nhất}
        \begin{itemize}
            \item Trong một đồ thị không trọng số, đường đi ngắn nhất giữa 2 đỉnh = min(số lượng cạnh phải đi qua).
            \item Trong đồ thị có trọng số, đường đi ngắn nhất là đường đi có tổng trọng số đạt min.
            \item Thuật toán Dijkstra là một thuật toán hiệu quả để giải quyết bài toán đường đi ngắn nhất trên đồ thị có trọng số.
        \end{itemize}
    \end{frame}

    \begin{frame}[t]
        \frametitle{Chu trình âm}
        \begin{itemize}
            \item Một chu trình trong đồ thị có độ dài âm được gọi là \textit{chu trình âm}.
            \item Một nhược điểm của thuật toán Dijkstra: lỗi khi đồ thị có chu trình âm!
            \item Để khắc phục, ta phải sử dụng một thuật toán khác: Bellman-Ford hoặc Floyd-Warshall.
        \end{itemize}
    \end{frame}

    \begin{frame}[t]
        \frametitle{Giới thiệu thuật toán Floyd-Warshall}
        \section{Giới thiệu}

        \begin{itemize}
            \item Được phát biểu bởi Robert W. Floyd và Stephen Warshall năm 1962.
            \item Là một thuật toán Quy hoạch động tiêu biểu.
        \end{itemize}
    \end{frame}

    \begin{frame}[t]
        \frametitle{Ý tưởng}
        \framesubtitle{Công thức truy hồi \& thuật toán}
        \section{Ý tưởng}
        \begin{itemize}
            \item Từ một đỉnh A, tìm đường đi đến tất cả các đỉnh còn lại.
            \item Từ một đỉnh B kề A, tìm đường đi đến tất cả các đỉnh còn lại.
        \end{itemize}

    \end{frame}

    \begin{frame}
        \frametitle{Ý tưởng (cont.)}
        \framesubtitle{Nhận xét}
        \textbf{Nhận xét}: Gọi F là mảng sau khi đã tính toán bằng thuật toán Floyd-Warshall.\pause
        \begin{itemize}
            \item Thuật toán Floyd-Warshall tính toán đường đi giữa tất cả các cặp đỉnh. Khi cần sử dụng, ta chỉ cần gọi F[<đỉnh đi>][<đỉnh đến>].\pause
            \item Thuật toán Floyd-Warshall còn có thể sử dụng để nhận biết chu trình âm: $F[A][B] < 0 \iff $ giữa A và B tồn tại một chu trình âm.
        \end{itemize}
    \end{frame}

    \begin{frame}[t]
        \frametitle{Cài đặt}
        \section{Cài đặt}
        while (a not sorted):
            generate a permutation from a.
    \end{frame}

    \begin{frame}[t]
        \section{Đánh giá}
        \frametitle{Đánh giá}
        \framesubtitle{Độ phức tạp thời gian}
        \begin{itemize}
            \item Độ phức tạp trong trường hợp tốt nhất: $\mathcal{O}(n^3)$
            \item Độ phức tạp trong trường hợp tệ nhất: $\mathcal{O}(n^3)$
            \item Độ phức tạp trung bình: $\mathcal{O}(n^3)$
        \end{itemize}
        \textbf{Nhận xét:} Thuật toán Floyd-Warshall có độ phức tạp khá \textbf{tệ} nhưng độ phức tạp mỗi truy vấn đường đi ngắn nhất sẽ là $\mathcal{O}(1)$.
    \end{frame}

    \begin{frame}[t]
        \frametitle{Đánh giá (cont.)}
        \framesubtitle{Độ phức tạp không gian}
        \begin{itemize}
            \item Độ phức tạp trong trường hợp tốt nhất: $\mathcal{O}(n^2)$
            \item Độ phức tạp trong trường hợp tệ nhất: $\mathcal{O}(n^2)$
            \item Độ phức tạp trung bình: $\mathcal{O}(n^2)$
        \end{itemize}
    \end{frame}

    \begin{frame}
        \frametitle{Tổng kết}
        \begin{itemize}
            \item Với độ phức tạp trung bình và tệ nhất đều là $\mathcal{O}(n^3)$, thuật toán Floyd nên được cân nhắc khi sử dụng trong các bài toán đồ thị.
            \item Có một cải tiến của thuật toán Floyd khiến độ phức tạp trung bình giảm từ $\mathcal{O}(n^3)$ xuống còn $\mathcal{O}(n^{3 - \epsilon})$
        \end{itemize}
    \end{frame}

    \begin{frame}
        \frametitle{Cải tiến thuật toán Floyd-Warshall}
        Một cải tiến được H.Grag và P.Rawat đưa ra năm 2012\cite{improve:article}, theo đó
    \end{frame}

    \begin{frame}
        \Large \centering
        The End!
    \end{frame}

    \begin{frame}[t]
        \frametitle{Tài liệu tham khảo}
        \section{Tài liệu tham khảo}
        \bibliographystyle{apalike}
        \bibliography{bib/algo}
    \end{frame}


\end{document}